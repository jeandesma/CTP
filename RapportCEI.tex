%% Based on a TeXnicCenter-Template by Gyorgy SZEIDL.
%%%%%%%%%%%%%%%%%%%%%%%%%%%%%%%%%%%%%%%%%%%%%%%%%%%%%%%%%%%%%

%------------------------------------------------------------
%
\documentclass[a4paper, 10pt]{article}

%%%%%%%%%%%%%%PACKAGES%%%%%%%%%%%%%%%%%%%%%
%%%%%%%%%%%%%%%%%%%%%%%%%%%%%%%%%%%%%%%%%%%
%%%%%%%%%%%%%%%%%%%%%%%%%%%%%%%%%%%%%%%%%%%

%
\usepackage{amsmath}%
\usepackage{amsfonts}%
\usepackage{amssymb}%
\usepackage{dsfont}
%\usepackage{graphicx}
\usepackage[latin1]{inputenc} 
%\usepackage[T1]{fontenc} 
%\usepackage{layout}
%\usepackage{setspace}
\usepackage{tikz}
%\usepackage{algorithm2e}
\usepackage[left=2.8cm,right=2.8cm,top=2cm,bottom=3cm]{geometry}
\usepackage{subcaption}

\usetikzlibrary{decorations}
\usetikzlibrary{decorations.pathmorphing}
\usetikzlibrary{decorations.pathreplacing}
\usetikzlibrary{decorations.shapes}
\usetikzlibrary{decorations.text}
\usetikzlibrary{decorations.markings}
\usetikzlibrary{decorations.fractals}
\usetikzlibrary{decorations.footprints}
%-------------------------------------------
%%%%%%%%%%%%%%THEOREMS%%%%%%%%%%%%%%%%%%%%%
%%%%%%%%%%%%%%%%%%%%%%%%%%%%%%%%%%%%%%%%%%%
%%%%%%%%%%%%%%%%%%%%%%%%%%%%%%%%%%%%%%%%%%%

\newtheorem{theorem}{Theorem}
\newtheorem{definition}{Definition}
\newtheorem{lemma}{Lemma}
\newtheorem{example}{Example}
\newtheorem{corollary}{Corollary}
\newtheorem{problem}{Problem}
\newtheorem{acknowledgement}[theorem]{Acknowledgement}
\newtheorem{axiom}[theorem]{Axiom}
\newtheorem{case}[theorem]{Case}
\newtheorem{claim}[theorem]{Claim}
\newtheorem{conclusion}[theorem]{Conclusion}
\newtheorem{condition}[theorem]{Condition}
\newtheorem{conjecture}[theorem]{Conjecture}

\newtheorem{criterion}[theorem]{Criterion}
\newtheorem{exercise}[theorem]{Exercise}
\newtheorem{notation}[theorem]{Notation}
\newtheorem{proposition}[theorem]{Proposition}
\newtheorem{remark}[theorem]{Remark}
\newtheorem{solution}[theorem]{Solution}
\newtheorem{summary}[theorem]{Summary}
\newenvironment{proof}[1][Proof]{\textbf{#1.} }{\ \rule{0.5em}{0.5em}}

%%%%%%%%%%%%%%COMMANDS%%%%%%%%%%%%%%%%%%%%%
%%%%%%%%%%%%%%%%%%%%%%%%%%%%%%%%%%%%%%%%%%%
%%%%%%%%%%%%%%%%%%%%%%%%%%%%%%%%%%%%%%%%%%%

\newcommand{\kctp}{$k$-CTP}
\newcommand{\set}[1]{\left\{ #1 \right\}}
\newcommand{\card}[1]{\left| #1 \right|}
\newcommand{\ith}[1]{#1^{\mbox{\scriptsize{th}}}}
\newcommand{\stpath}{$(s,t)$-path}
\newcommand{\stpaths}{$(s,t)$-paths}
\newcommand{\omegamin}{\omega_{\mbox{\scriptsize{min}}}}
%Grandes lettres italiques
\newcommand{\mcalp}{\mathcal{P}}
\newcommand{\mcals}{\mathcal{S}}
\newcommand{\mcale}{\mathcal{E}}
\newcommand{\mcalb}{\mathcal{B}}
\newcommand{\mcall}{\mathcal{L}}
\newcommand{\mcalr}{\mathcal{R}}
\newcommand{\mcalt}{\mathcal{T}}


%%%%%%%%%%%%%%DOCUMENT%%%%%%%%%%%%%%%%%%%%%
%%%%%%%%%%%%%%%%%%%%%%%%%%%%%%%%%%%%%%%%%%%
%%%%%%%%%%%%%%%%%%%%%%%%%%%%%%%%%%%%%%%%%%%

\begin{document}

\title{Canadian travellers minimize the traversed distance: definitions, bounds and heuristics}

\maketitle

%\begin{abstract}
%%
%\end{abstract}


%\smallskip
%\begin{center}
%\noindent \textbf{Keywords.} blabla. 
%\end{center}
%\smallskip


\section{Introduction}

The \textit{Canadian Traveller Problem} (CTP) was introduced by Papadimitriou and Yannakakis~\cite{PaYa91}. This is a generalization of the \textit{Shortest Path Problem}. Given an undirected weighted graph $G=(V,E,\omega)$ and two nodes $s,t \in V$, the objective is to design a strategy in order to make a traveller walk from $s$ to $t$ through the graph $G$, knowing that some edges can be blocked. The traveller ignores which edges are blocked when he begins and discover them when he visits an adjacent node. The $k$-\textit{Canadian Traveller Problem} (\kctp) is the parameterized variant of CTP where we specify an upper bound for the total number of blocked edges. Both CTP and \kctp ~are PSPACE-complete~\cite{BaSc91,PaYa91}.

\paragraph{State-of-the-art:}Several strategies have been designed and studied through the competitive analysis, which is a way to assess the quality of an online algorithm. A first class of strategies to be considered are deterministic. Westphal proved that there is no deterministic algorithm that can achieve a better competitive ratio than $2k+1$ where $k$ is an upper bound of the number of blockages and that this ratio is achieved by \textsc{reposition} algorithm~\cite{We08}. However, in practice (for example in the case of an urban network), returning to node $s$ everytime the traveller is blocked does not seem to be an efficient strategy. This is why Xu et al. introduced the \textsc{greedy} algorithm for the CTP which achieves a $2^{k+1}-1$ ratio~\cite{XuHuSuZh09}. For grids, they showed that the \textsc{greedy} strategy achieves a $\mathcal{O}\left(1\right)$ ratio, independent of $k$, under realistic hypotheses. Both \textsc{greedy} and \textsc{reposition} strategies are executed in polynomial time.

A second class of strategies are randomized. We evaluate these strategies by supposing that an oblivious adversary is setting the blocked edges. Westphal proved that there is no randomized algorithm that can achieve a ratio lower than $k+1$~\cite{We08}. Recently, two randomized algorithms have been proposed. Demaine et al. designed a strategy for arbitrary graphs with a ratio $\left(1+\frac{\sqrt{2}}{2}\right)k+1$. This is executed in time $\mathcal{O}\left(k\mu^2\card{E}^2\right)$ where $\mu$ is an parameter that can potentially be exponential~\cite{DeHuLiSa14}. However, for a rather large class of graph (graphs with a reasonable number of groups of path, where a group of path is a set of paths which have the same length), this strategy offers better results than deterministic methods and is executed in polynomial time. Furthermore, Bender et al. studied the specific case of node-disjoint-paths graphs and proposed a strategy of ratio $\left(k+1\right)$ and with a polynomial running time for this kind of graphs~\cite{BeWe15}. Their method consists in a randomized \textsc{reposition}: we assign a probability $p_i$ to each path $P_i$ and execute a draw: the traveller crosses the path and, if he is blocked, returns to $s$ and restarts the process.

% \paragraph{Contributions:} The main objective of this study is to design a randomized strategy which has a $(K+1)$-competitive ratio for graphs $G$. Here is a list of our contributions:
% \begin{itemize}
% \item We introduce the \knctp ~which the two-parameters variant of CTP where $K$ is the upper bounded on the number of blocked edges and $n$ is the exact number of simple \stpaths.
% \item We prove that the competitive ratio of the randomized reposition strategy on $K$-CTP, noted $\arb$, is at least $2k+1$ for any kind of graphs. It implies that this method is specific for a certain class of graphs, containing at least node-disjoint paths instances.
% \item We propose an optimal randomized strategy to treat graphs which fulfils the following condition: there is a $\left(K+1\right)$-partition $\mcalp_1 \cup \ldots \cup \mcalp_{K+1}$ of the set of simple \stpaths ~such that, for any path $P_i \in \mcalp_i$ ,$P_j \in \mcalp_j$, $j\neq i$, $P_i$ and $P_j$ are node-disjoint. Its competitive ratio is $K+1$.
% \item We build a randomized strategy called the Canadian Traveller Random Search (CTRS) for graphs such that simple \stpaths ~have the same cost. We note it $\ars$. The competitive ratio is $K+1$. This randomized strategy is optimal for this kind of instances.
% \end{itemize}

\section{Preliminaries}

\subsection{Notations and definitions with a single traveller} 
The traveller traverses an undirected weighted graph $G=\left(V,E,\omega\right)$, $n = \card{V}$ and $m = \card{E}$. He starts his walk at source $s \in V$. His objective is to reach target $t\in V$ with a minimum cost (also called distance), which is the sum of the weights of edges traversed. Set $E_*$ contains blocked edges, which means that when the traveller reaches an endpoint of one of these edges, he discovers that he cannot pass through it. A pair $\left(G,E_*\right)$ is called a \textit{road map}. From now on, we suppose that any road map $\left(G,E_*\right)$ is feasible, {\em i.e.} $s$ and $t$ are always connected in graph $G\backslash E_*$.

We remind the definition of the competitive ratio introduced in \cite{BoEl98}. Let $\omega_A\left(G,E_*\right)$ be the distance traversed by the traveller guided by a given strategy $A$ on graph $G$ from source $s$ to target $t$ with blocked edges $E_*$. The shortest \stpath ~in $G\backslash E_*$ is called the \textit{optimal offline path} of map $\left(G,E_*\right)$ and its cost, noted $\omegamin\left(G,E_*\right)$, is the optimal offline cost of map $\left(G,E_*\right)$. Strategy $A$ is $c_A$-competitive if, for any road map $\left(G,E_*\right)$:

\[
\omega_A\left(G,E_*\right) \leq c_A\omegamin\left(G,E_*\right) + \eta,
\]
%\label{eq:deter-comp}
where $\eta$ is constant. For randomized strategies, it becomes:

\[
\mathbb{E}\left[\omega_A\left(G,E_*\right)\right] \leq c_A\omegamin\left(G,E_*\right) + \eta.
\]

\subsection{Notations and definitions with $L$ travellers}

To be completed...

\section{Bounds of competitiveness with multiple travellers}
We consider the following deterministic and online algorithm for K blocked edges and L travellers with complete communication: we launch the travellers one at a time. If the traveller meets a blocked edge, it has to stay there and we launch another traveller who will have to take the second shortest path possible. Once again when this one meets a blocked edge, it stays there and another is launched. Let's call this strategy the abandonment strategy.

\paragraph{Lemma 3.1:} The cost of the abandonment strategy on the Westphale graph is : K + 1 if L $>$ K and 2K - L + 1 if L $<$ K.

\paragraph{Proof:} Preuve Aurelie.

\paragraph{Lemma 3.2:} The abandonment strategy has the optimal cost on the Westphal graph.

\paragraph{Proof:} This Lemma can be proved using recurrence.  

\paragraph{Recurrence hypothesis:} For K (blocked edges), for all L the optimal cost is : K + 1 if L $>$ K and 2K - L + 1 if L $<$ K.

\paragraph{Initialization:} Let's take the example of 1 traveller. As shown in the existing literature we already have the cost equal to K + 1.

\paragraph{Recurrence:} We know that there is a natural integer n number of travellers for which the cost is: if n $>$ K, the cost is K + 1, and if n $<$ K, the cost is 2K - L + 1. Let's prove it for n + 1. 
If n + 1 $<$ K, since we are in the worst case scenario, the n travellers have to discover the K blocked edges before reaching the end. To do so, the cost is at least of K. Then 1 is added to reach the end. However, our strategy has a cost of K + 1 also, therefore once again, for n + 1 $<$ K, the optimal cost is K + 1. 

Now if n + 1 $>$ K, we launch the first traveller. He discovers one blocked edge and stays there. We are now facing a problem of n travellers facing K-1 blockage. By recurrence hypothesis we know that for that case the cost is 2(K-1) - n + 1 because n $>$ K - 1. To that cost we add the cost of discovering the first blocked edge. We have 2(K + 1) - n + 2 which is equal to 2K - (n + 1) - 1. This shows the second part of the recurrence.

We can therefore conclude that the optimal cost for the Westphal graph is : K + 1 if L $>$ K and 2K - L + 1 if L $<$ K. This is the cost of the abandonment strategy. Then this proves Lemma 3.2.


\section{Bounds of competitiveness with multiple travellers: randomized algorithms}

% K blocked edges, L travellers, complete communication
Now we consider the randomized online algorithms for $K$ blocked edges and $L$ travellers with complete communication: we assign a probability $pi$ to each path $Pi$ and execute a draw: let the travellers crosses the path one by one, if the first traveller is blocked, leave him at the blocked point and let the second traveller to start the process; if all the travellers are blocked, then let the first traveller return to $s$ and restarted the process. 
We can get:

\begin{lemma}
If we have more travellers than blocked edges, there is no randomized online algorithm with competitive ratio less than:
\[
\frac{k+2}{2}
\]
\label{lemma_moretr}
\end{lemma}

\begin{proof} 
With $K$ smaller than $L$, it is certain that we do not have to let a traveller return to s before we find all the blocked edges. If the algorithm is successful on its $(l+1)^{\text{th}}$ try, only the $(l+1)^{\text{th}}$ try cost ($1+ \eta$) and so the total cost will be:
\[
(l-1)\cdot 1 + (1+\eta) = l+\eta
\]
and the probability will be: 
\[
\frac{1}{l+1}
\]
Hence, its expected cost is at least:
\begin{displaymath}
\sum_{l=1}^{k+1}(l+\eta)\cdot \frac{1}{l+1} = \frac{K+2}{2}+\eta
\end{displaymath}
As we know, the expected optimal offline cost is $1 + \eta$, so the competitive ratio is 
\[
\frac{K+2}{2}.
\]
\end{proof}

\begin{lemma}
If we have less travellers than blocked edges $ (K <  L )$, there is no randomized online algorithm with competitive ratio less than:
\[
K+2-L+ \frac{L(L-1)}{2(K+1)}
\]
\end{lemma}

\begin{proof}
Similar with the proof of ~\ref{lemma_moretr}, if the algorithm is successful on its $(l+1)^{\text{th}}$ try, the probability will be: 
\[
\frac{1}{l+1}
\]
But to calculus the cost, we have to consider two possibilities: 
if $l < L$, which means we have find all the blocked edges without letting any traveller to return to $s$, the cost will be 
\[
(l-1)\cdot1 + 1 +\eta = l + \eta;
\]
if $l > L$, it means that we have not find all the blocked edges after letting all the travellers depart $s$,  so we have to let $(l-L)$ travellers return to $s$ and restart the process. In this case, there will be $L$ travellers with cost 1, $(l-L-1)$ travellers with cost 2 and the last traveller with cost $(2+\eta)$. So the total cost will be:
\[
L\cdot 1 + (l-L-1)\cdot2 + 2 +\eta = 2l - L + \eta .
\]
So, its expected cost is at least:
\[
\sum_{l=1}^{L}(l+\eta)\cdot \frac{1}{l+1}  +  \sum_{L+1}^{K+1}(2l-L+\eta)\cdot \frac{1}{l+1}
 =K+2-L+ \frac{L(L-1)}{2(K+1)} + \eta
\]
As we know, the expected optimal off-line cost is $1 + \eta$, so the competitive ratio is 
\[
K+2-L+ \frac{L(L-1)}{2(K+1)}.
\]
\end{proof}

If we consider the special case $L=1$ which means only one traveller, the competitive ratio becomes:
\[
K+2-L+ \frac{L(L-1)}{2(K+1)} = K+2-1 = K+1
\]
This result corresponds to the randomized case of Bender et al.~\cite{BeWe15}.

\bibliographystyle{plain}
\bibliography{ctp.bib}

\end{document}